\documentclass[11pt]{article}

% Change "review" to "final" to generate the final (sometimes called camera-ready) version.
% Change to "preprint" to generate a non-anonymous version with page numbers.
\usepackage[numbers]{natbib}
\usepackage[]{acl}
\usepackage{times}
\usepackage{latexsym}

% For proper rendering and hyphenation of words containing Latin characters (including in bib files)
\usepackage[T1]{fontenc}

% This assumes your files are encoded as UTF8
\usepackage[utf8]{inputenc}
\usepackage{microtype}
\usepackage{inconsolata}
\usepackage{graphicx}


\title{Group 73 Progress Report:\\Wildfire Detection Classification Plan}


\author{Andy Huynh, Berk Yilmaz, Tanisha Tasnin \\
  \textbf{huynha3@mcmaster.ca}, \textbf{yilmag1@mcmaster.ca},  \textbf{tasnint@mcmaster.ca}}

\begin{document}
\maketitle

\section{Introduction}

% Here, write a brief introduction to the problem you are solving. This can be adapted from your problem description and motivation from the original proposal. This should be around 0.25-0.5 pages.
% TODO: Replace [] with citations {\cite{keyname}}
Wildfires are becoming increasingly prevalent and devastating due to climate change \cite{ref1_WildfiresClimateChange2025}. 
This poses a growing threat to both human safety and the environment, causing billions of dollars in damages each year and severely affecting air quality and biodiversity \cite{ref2_WildfireSmoke2024}, \cite{ref3_WildfireTrends2026}, \cite{ref4_WildfireEconomicToll2025}.  
Early detection of wildfires is crucial for effective response and mitigation efforts. However, detecting wildfires at an early stage can be challenging.
Subtle indicators such as light smoke or small flames can be obscured by dense vegetation, clouds, or varying terrain, often making traditional monitoring methods unreliable \cite{ref5_ozel2024forestfire}.
In an attempt to address this issue, our project explores the use of Convolutional Neural Networks (CNNs) for automatic wildfire detection from images \cite{ref6_tsalera2023realTimeFireDetection}, \cite{ref7_spiller2023wildfire_prisma}.
Our project formulates the task as a binary classification problem: given an RGB image of an area, the model identifies it as a “fire” or “no fire” scenario using ‘The wildfire dataset’ by El Madafri on kaggle \cite{ref8_elmadafri2023wildfiredataset}.

% Here, talk about the related work you encountered for your approach. Cite at least 5 references. Refer to item 2. No one has done exactly your task? Write about the most similar thing you can find. This should be around 0.25-0.5 pages.
\section{Related Work}
Advancements in wildfire research have led to the use of deep learning to enhance prediction, monitoring, and detection capabilities []. 
There are several existing solutions that use a combination of various CNN architectures and other neural networks for wildfire prediction. 
Popular models include \textbf{FirePred} and \textbf{WFNet}. 
\textit{FirePred} is a hybrid multi-temporal CNN model for wildfire spread prediction and \textit{WFNet} is a hierarchicalCNN for wildfire spread prediction [], []. 
However these models, like most renowned ones, focus on modeling wildfire spread and spatiotemporal dynamics rather than direct detection from visual data which our project addresses. 
For direct wildfire detection from images, the paper \textbf{Advanced Wildfire Detection Using Deep Learning Algorithms: A Comparative Study of CNN Variants} is a notable example []. 
This study evaluates models such as InceptionV3, Xception, and NASNetMobile on over 25,000 images, achieving accuracies above 98\%. 
While this study focuses on benchmarking CNN architectures for accuracy, our project differs by optimizing image preprocessing and augmentation pipelines to balance accuracy with computational efficiency for real-time detection. 
There are however other CNN projects such as Malaria Detection using TensorFlow's malaria dataset that resemble our project's workflow more closely [], []. 
Similar to our approach, they emphasize image preprocessing, model training, and performance optimization for efficient binary classification.

% You should write about your dataset here, following the guidelines regarding item 1. This section may be 0.5-1 pages. Depending on your specific dataset, you may want to include subsections for the preprocessing, annotation, etc.
\section{Dataset}
We are training our model with the splits gathered from Kaggle. We have 1887 data points for training, 402 for validation, and 410 for test. This roughly corresponds to a 70-15-15 split of our dataset.

The dataset is composed of images mainly depicting forests, fields, and rural areas. Images are sorted into two categories; ``fire'' and ``nofire''. These categories constitute the labels we use during training.

The images are of different resolutions and quality. Some images are considerably higher resolution than others, and it is apparent that the images are from different time periods and were taken by different devices. This presents a minor challenge in training our model, where we need to ensure our model can use each and every one of these data points. We tacke this by normalizing our data as outlined in the Features section.


% Describe any features you used for your model, or how your data was input to your model. Are you doing feature engineering or feature selection? Are you learning embeddings? Is it all part of one neural network? Refer to item 2. This may range from 0.25 pages to 0.5 pages.
\section{Features}
The feature set for our model comprises of the pixel values from the images themselves. 
The images are mainly RGB type images meaning they compose of three colour channels red, green, and blue. 
The resolution also dictates the number of pixels within a given image. 
For example, a 128x128 resolution image means there are 128 * 128 * 3 = 49152 feature values that corresponde to
the RGB colour values. The higher the resolution more features there are and vice versa. 
During preprocesssing, the images are compressed to a fixed resolution (128x128, 224x224, etc...),
 so that the input array remains the same. The pixel values are placed in 3D-dimensional matrix.
This form of feature engineering alows for spacial recognition and patten extraction on a 3D space.                  

% Describe your model and implementation here. Refer to item 4. This may take around a page.
\section{Implementation}

Our model is a feedforward classification model. It is composed of TODO Conv2D layers of size TODO with ReLu activation, followed by a Dense layer with TODO units. The model has TODO parameters in total, taking up about TODO MB in total.

Before training, we optimized various parameters empirically. We set out certain augmentations and resolutions to be tested, then trained a model on every dimension of this space via nested for loops that can be found in our code. This process is further outlined in the Features section. The size of layers and the number of layers were determined by manual empirical testing.

% How are you evaluating your model? What results do you have so far? What are your baselines? Refer to item 5. This may take around 0.5 pages.
\section{Results and Evaluation}
Our model evaluation process focuses on comparing different image preprocessing and augmentation configurations to identify the combination that offers the best trade-off between computational efficiency and classification accuracy.
The array of pixel sizes we are testing includes pixel sizes, rotation, shifting height and width, zoom, flip, and brightness.
Each configuration was trained for five epochs, and both validation accuracy and average computation time per epoch were recorded to measure performance.
The system automatically stops iterating over new resolutions when accuracy improvements fall below 3\%, reducing unnecessary computation.

Initial experiments at lower resolutions (128x128) with standard augmentation (rotation, brightness, and zoom variations) achieved validation accuracies around 57 - 58\%.

Larger image resolutions are currently being evaluated to determine whether higher spatial detail leads to significant accuracy gains. The best configuration identified so far will be used to train the final CNN model, which is then validated and tested on separate data subsets.

The final architecture employs a four-block convolutional neural network (Conv2D-MaxPooling layers) with dropout for regularization, compiled with binary cross-entropy loss, a metric of accuracy, and the Adam optimizer. Model performance is monitored across epochs using accuracy and loss curves saved as visual outputs. Test performance will be reported once all preprocessing configurations finish executing. For baselines, the model without augmentation and at lower resolutions serves as the control, while subsequent tests compare the impact of increasing image size and data augmentation strength.

Our current best model uses no augmentation and normalizes images to 224x224. This model achieves a training accuracy of 83.73\% and a validation accuracy of 79.85\%. However, this model is an early experiment and many areas of improvement were raised during the experimentation process. as outlined in the Feedback and Plans section.

% To evaluate model performance, the following classification metrics are computed on the test set:
% \setlength{\tabcolsep}{2pt}
%
% >>>>>>> origin/tanisha-2
% \begin{table}[h!]
% \centering
% \begin{tabular}{lcccc}
% \hline
% \textbf{Metric} & \textbf{Fire} & \textbf{No Fire} & \textbf{Weighted Avg.} & \textbf{Support} \\
% \hline
% Precision & xxx & xxx & xxx & xxx \\
% Recall    & xxx & xxx & xxx & xxx \\
% F1-Score  & xxx & xxx & xxx & xxx \\
% Accuracy  & \multicolumn{4}{c}{xxx} \\
% \hline
% \end{tabular}
% \caption{Model performance metrics on the test set (placeholders to be updated after full training).}
% \label{tab:performance_metrics}
% \end{table}
%


% Write about your plans for the remainder of the project. This should include a discussion of the feedback you received from your TA, and how you plan to improve your approach. Reflect on your implementation and areas for improvement. Refer to item 6. This may be around 0.5 pages.
\section{Feedback and Plans}
The primary feedback focused on improving the reproducibility and interpretability of the results. 
Specifically, the TA recommended implementing a more consistent validation split across all preprocessing configurations to ensure that performance differences are not influenced by random sampling. 
Additionally, they advised that the baseline model should be clearly defined and quantitatively compared against augmented configurations to emphasize the measurable impact of each experimental change.
Another key piece of feedback was to include computational metrics—such as average training time per epoch and resource utilization—in the results table. 
This would clearly demonstrate trade-offs between model accuracy and computational efficiency, which is key to optimizing performance under limited resources.
The TA also suggested monitoring for potential overfitting by tracking training and validation accuracy curves more closely and introducing early stopping or dropout adjustments if the validation loss diverges.

For the remainder of the project, we plan to incorporate these recommendations by (1) locking a fixed random seed for reproducibility, (2) ensuring the dataset splits are stratified, (3) expanding the evaluation metrics to include F1-score and confusion matrices for a more detailed performance assessment, and (4) introducing a systematic summary table comparing all tested resolutions and augmentation levels.
We will also document the data preprocessing pipeline in greater detail to improve transparency and ensure that results are easily replicable. 
Finally, once the best-performing configuration is identified, we will retrain the model for additional epochs and evaluate it on a held-out test set to provide final quantitative results and visualizations.





% *****************************************************************************************
% OPTIONAL SECTION 8: TEMPLATE NOTES, REFERENCES, TABLES AND FIGURES, EQUATIONS
% \section{Template Notes}

You can remove this section or comment it out, as it only contains instructions for how to use this template. You may use subsections in your document as you find appropriate.

\subsection{Tables and figures}

See Table~\ref{citation-guide} for an example of a table and its caption.
See Figure~\ref{fig:experiments} for an example of a figure and its caption.


% \begin{figure}[t]
%   \includegraphics[width=\columnwidth]{example-image-golden}
%   \caption{A figure with a caption that runs for more than one line.
%     Example image is usually available through the \texttt{mwe} package
%     without even mentioning it in the preamble.}
%   \label{fig:experiments}
% \end{figure}

% \begin{figure*}[t]
%   \includegraphics[width=0.48\linewidth]{example-image-a} \hfill
%   \includegraphics[width=0.48\linewidth]{example-image-b}
%   \caption {A minimal working example to demonstrate how to place
%     two images side-by-side.}
% \end{figure*}


\subsection{Citations}

\begin{table*}
  \centering
  \begin{tabular}{lll}
    \hline
    \textbf{Output}           & \textbf{natbib command} & \textbf{ACL only command} \\
    \hline
    \citep{Gusfield:97}       & \verb|\citep|           &                           \\
    \citealp{Gusfield:97}     & \verb|\citealp|         &                           \\
    \citet{Gusfield:97}       & \verb|\citet|           &                           \\
    \citeyearpar{Gusfield:97} & \verb|\citeyearpar|     &                           \\
    \citeposs{Gusfield:97}    &                         & \verb|\citeposs|          \\
    \hline
  \end{tabular}
  \caption{\label{citation-guide}
    Citation commands supported by the style file.
  }
\end{table*}

Table~\ref{citation-guide} shows the syntax supported by the style files.
We encourage you to use the natbib styles.
You can use the command \verb|\citet| (cite in text) to get ``author (year)'' citations, like this citation to a paper by \citet{Gusfield:97}.
You can use the command \verb|\citep| (cite in parentheses) to get ``(author, year)'' citations \citep{Gusfield:97}.
You can use the command \verb|\citealp| (alternative cite without parentheses) to get ``author, year'' citations, which is useful for using citations within parentheses (e.g. \citealp{Gusfield:97}).

\subsection{References}

\nocite{Ando2005,andrew2007scalable,rasooli-tetrault-2015}

Many websites where you can find academic papers also allow you to export a bib file for citation or bib formatted entry. Copy this into the \texttt{custom.bib} and you will be able to cite the paper in the \LaTeX{}. You can remove the example entries.

\subsection{Equations}

An example equation is shown below:
\begin{equation}
  \label{eq:example}
  A = \pi r^2
\end{equation}

Labels for equation numbers, sections, subsections, figures and tables
are all defined with the \verb|\label{label}| command and cross references
to them are made with the \verb|\ref{label}| command.
This an example cross-reference to Equation~\ref{eq:example}. You can also write equations inline, like this: $A=\pi r^2$.


% \section*{Limitations}
% *****************************************************************************************


% Write in this section a few sentences describing the contributions of each team member. What did each member work on? Refer to item 7.
\section*{Team Contributions}
Andy Huynh contributed by writing the Features and Implementations sections (4 \& 5) of the report. 
He provide some early implementation for the code, as well look over made additions to sections 1,2,3,6, and 7. 


% Bibliography entries for the entire Anthology, followed by custom entries
%\bibliography{custom,anthology-overleaf-1,anthology-overleaf-2}
% Custom bibliography entries only
\bibliographystyle{acl_natbib}
\bibliography{custom}

% \appendix
% \section{Example Appendix}
% \label{sec:appendix}
% This is an appendix.

\end{document}
