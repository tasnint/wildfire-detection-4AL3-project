Advancements in wildfire research have led to the use of deep learning to enhance prediction, monitoring, and detection capabilities []. 
There are several existing solutions that use a combination of various CNN architectures and other neural networks for wildfire prediction. 
Popular models include \textbf{FirePred} and \textbf{WFNet}. 
\textit{FirePred} is a hybrid multi-temporal CNN model for wildfire spread prediction and \textit{WFNet} is a hierarchicalCNN for wildfire spread prediction [], []. 
However these models, like most renowned ones, focus on modeling wildfire spread and spatiotemporal dynamics rather than direct detection from visual data which our project addresses. 
For direct wildfire detection from images, the paper \textbf{Advanced Wildfire Detection Using Deep Learning Algorithms: A Comparative Study of CNN Variants} is a notable example []. 
This study evaluates models such as InceptionV3, Xception, and NASNetMobile on over 25,000 images, achieving accuracies above 98\%. 
While this study focuses on benchmarking CNN architectures for accuracy, our project differs by optimizing image preprocessing and augmentation pipelines to balance accuracy with computational efficiency for real-time detection. 
There are however other CNN projects such as Malaria Detection using TensorFlow's malaria dataset that resemble our project's workflow more closely [], []. 
Similar to our approach, they emphasize image preprocessing, model training, and performance optimization for efficient binary classification.