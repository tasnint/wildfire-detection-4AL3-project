Our model evaluation process focuses on comparing different image preprocessing and augmentation configurations to identify the combination that offers the best trade-off between computational efficiency and classification accuracy.
The array of pixel sizes we are testing includes pixel sizes, rotation, shifting height and width, zoom, flip, and brightness.
Each configuration was trained for five epochs, and both validation accuracy and average computation time per epoch were recorded to measure performance.
The system automatically stops iterating over new resolutions when accuracy improvements fall below 3\%, reducing unnecessary computation.

Initial experiments at lower resolutions (128×128) with standard augmentation (rotation, brightness, and zoom variations) achieved validation accuracies around 57 - 58\%.

Larger image resolutions improved the accuracy with higher spatial detail, leading to significant accuracy gains. The best configuration was identified during training for the final CNN model as the standard training augmentation, which was then validated and tested on separate data subsets. 
The final architecture employs a four-block convolutional neural network (Conv2D-MaxPooling layers) with dropout for regularization, compiled with binary cross-entropy loss, a metric of accuracy, and the Adam optimizer. Model performance is monitored across epochs using accuracy and loss curves saved as visual outputs. Test performance will be reported once all preprocessing configurations finish executing. For baselines, the model without augmentation and at lower resolutions serves as the control, while subsequent tests compare the impact of increasing image size and data augmentation strength.

Our current best model uses the standard augmentation (minor changes) and normalizes images to 224×224. This model achieves a training accuracy of 83.73\% and a validation accuracy of 79.85\% as seen in Figure~\ref{fig:training_curve}, which is just below standard good model averages; however, this model is an early experiment, and many areas of improvement were raised during the experimentation process as outlined in the Feedback and Plans section.
For Figures~\ref{fig:training_curve} and \ref{fig:loss_curve}, the training and validation curves converge around 13-16+ Epochs, meaning further Epochs could lead to strong overfitting. There is some instability and various spikes on curves, which means the model needs some improvement in terms of hyperparameter tuning, more test samples, or batch sizes.
As shown in Tables~\ref{tab:classification_report} and \ref{tab:performance_metrics}, precision, specificity, and overall performance hoevers arounds 72-88\% with the main weakness being a relative high false negative rate shown in the confusion matrix in Figure~\ref{fig:confusion_matrix} that needs to be addresss majory in order to this model to be viable in identifying real fires.

% Based on the confusion and the proceeding training and loss curves over epochs

\begin{figure}[htbp]
    \centering
    \includegraphics[width=\columnwidth]{results/reproduced_confusion_matrix.png}
    % NOT: width=\textwidth (this spans both columns)
    % USE: width=\columnwidth (fits in one column)
    \caption{Confusion matrix on test set (n=410). The model achieved 197 true positives (TP), 140 true negatives (TN), 19 false positives (FP), and 54 false negatives (FN), resulting in 82.2\% overall accuracy.}
    \label{fig:confusion_matrix}
\end{figure}

% \begin{figure*}[t]
%     \centering
%     \includegraphics[width=0.85\textwidth]{models/training_curves_best.png}
%     \caption{Training and validation metrics over 20 epochs...}
%     \label{fig:training_curves}
% \end{figure*}

\begin{figure}[htbp]
    \centering
    \includegraphics[width=\columnwidth]{results/accuracy_training_curves.png}
    % NOT: width=\textwidth (this spans both columns)
    % USE: width=\columnwidth (fits in one column)
    \caption{Graph showing the training and validation accuracty over Epochs}
    \label{fig:training_curve}
\end{figure}

\begin{figure}[htbp]
    \centering
    \includegraphics[width=\columnwidth]{results/loss_training_curves.png}
    % NOT: width=\textwidth (this spans both columns)
    % USE: width=\columnwidth (fits in one column)
    \caption{Graphing showing the train and validation loss over Epochs}
    \label{fig:loss_curve}
\end{figure}

% To evaluate model performance, the following classification metrics are computed on the test set:
% \setlength{\tabcolsep}{2pt}
%
% >>>>>>> origin/tanisha-2
% \begin{table}[h!]
% \centering
% \begin{tabular}{lcccc}
% \hline
% \textbf{Metric} & \textbf{Fire} & \textbf{No Fire} & \textbf{Weighted Avg.} & \textbf{Support} \\
% \hline
% Precision & xxx & xxx & xxx & xxx \\
% Recall    & xxx & xxx & xxx & xxx \\
% F1-Score  & xxx & xxx & xxx & xxx \\
% Accuracy  & \multicolumn{4}{c}{xxx} \\
% \hline
% \end{tabular}
% \caption{Model performance metrics on the test set (placeholders to be updated after full training).}
% \label{tab:performance_metrics}
% \end{table}
%
