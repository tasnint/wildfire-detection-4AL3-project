Wildfires are becoming increasingly prevalent and destructive due climate change, particularly in northern and temperate forest regions such as Canada \cite{ref1_WildfiresClimateChange2025}. 
This poses a growing threat to both human safety and the environment with upto 2740 deaths in Canada between 2013 to 2018 alone \cite{ref2_WildfireSmoke2024}. 
The economic consequences are equally substantial: a good example being the 2020 Australian mega-fires which caused approximately US \$20 billion in damages \cite{ref3_WildfireTrends2026}. 
Wildfires also have profound and well-documented impacts on air quality and biodiversity worldwide. %\cite{ref4_WildfireBiodiversity}.  
Early detection of wildfires is crucial for effective response and mitigation efforts. 
However, detecting wildfires at an early stage can be challenging.
Subtle indicators such as light smoke or small flames can be obscured by dense vegetation, clouds, or varying terrain, often making traditional monitoring methods unreliable \cite{ref5_ozel2024forestfire}.
In an attempt to address this issue, our project explores the use of Convolutional Neural Networks (CNNs) for automatic wildfire detection from images. 
Prior work by Tsalera et al. and Spiller et al. provided valuable insights for the research going into our project, highlighting both foundational and advanced CNN-based wildfire detection techniques \cite{ref6_tsalera2023realTimeFireDetection}, \cite{ref7_spiller2023wildfire_prisma}.
Our project formulates the task as a binary classification problem: given an RGB image of an area, the model identifies it as a “fire” or “no fire” scenario using ‘The wildfire dataset’ by El Madafri on kaggle \cite{ref8_elmadafri2023wildfiredataset}.
To benchmark performance, we begin with a conventional CNN serving as our baseline. 
We then created 3 separate and individual models that used modern architectures: ResNet18, MobileNetV2, and EfficientNet-B0.
These model choices were motivated by influential works from He et al., Tan et al., and Howard et al. %\cite{he2015deep}, \cite{tan2019efficientnet}, \cite{howard2017mobilenets}



